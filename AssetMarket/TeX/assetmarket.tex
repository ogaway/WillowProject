\documentclass[dvipdfmx,12pt]{beamer}
%\documentclass[dvipdfmx,12pt,handout]{beamer}
\usepackage{pxjahyper}
\usepackage{minijs}
\usepackage{otf}
\renewcommand{\kanjifamilydefault}{\gtdefault}
\useoutertheme{infolines}
\usecolortheme[RGB={0,128,0}]{structure}
\usecolortheme{dolphin}
\setbeamertemplate{navigation symbols}{}
\setbeamercolor{frametitle}{fg=white, bg=gray}

\title{Asset Market}
\author{小川慶将}
\date{\today}

\begin{document}
\begin{frame}\frametitle{}
 \titlepage
\end{frame}

\begin{frame}\frametitle{発表の流れ}
 \tableofcontents
\end{frame}

\section{株価の決定メカニズム}
\subsection{効率的市場か?バブルか?}
\begin{frame}
\frametitle{効率的市場か?バブルか?}
\begin{itemize}\setlength{\parskip}{0.5em}
\item
効率的市場仮説:\\
株価は株式がもつ本来の価値、すなわちファンダメンタルバリューに等しくなる。\\
ファンダメンタルバリューとは、株式が将来にわたって生み出す配当の価値をすべて合計したもの。
\item
バブル説:\\
株価は投資家の心理的なものによっていい加減に決まる。人々が上がると思えば上がるし、下がると思えば下がる。
\item
どっちが正しいのか?\\
→現実の市場ではファンダメンタルバリューを正確に知ることが出来ないので解決困難。
\end{itemize}
\end{frame}

\section{実験結果考察}
\subsection{先行研究}
\begin{frame}
\frametitle{先行研究}
\begin{itemize}\setlength{\parskip}{0.5em}
\item
Smith, Suchanek and Williamsによる先駆的実験。\\
おそらく今日の実験とほぼ同じもの。
\begin{figure}
\centering
\includegraphics[width=50mm]{smith.pdf}
\caption{先行研究での実験結果}
\label{fig:bubble}
\end{figure}
\end{itemize}
\end{frame}

\section{なぜバブルは生じたのか?}
\subsection{バブルの原因とは}
\begin{frame}
\frametitle{バブルの原因とは}
\begin{itemize}\setlength{\parskip}{0.5em}
\item
非合理バブル:\\
投資家がファンダメンタルバリューを理解していなかったことにより生じたバブル。。
\begin{itemize}\setlength{\parskip}{0.5em}
\item
1期あたりの期待値が24になること
\item
ファンダメンタルバリューが24×(残りの配当回数)となること
\end{itemize}
\item
投機的バブル:\\
投資家が株の値上がり益を狙って生じたバブル。
\end{itemize}
\end{frame}


\subsection{非合理バブル}
\begin{frame}
\frametitle{非合理バブル}
\begin{itemize}\setlength{\parskip}{0.5em}
\item
ファンダメンタルバリューの計算の困難さの程度の異なる3種類の実験を行う。\\
\begin{itemize}\setlength{\parskip}{0.5em}
\item
今日やった実験と同じく、毎期末に配当を支払う実験
\item
毎期末に支払う額を小さくして、最終期末に大きな配当を支払う実験
\item
最終期末にのみ配当を支払う実験
\end{itemize}
\item
バブルが発生する確率は3つ目の実験が1番少なかった。
\item
現実の株式市場でのファンダメンタルバリューの計算はさらに複雑なので、投資家の非合理性によってバブルが発生している可能性あり。
\end{itemize}
\end{frame}


\begin{frame}
\frametitle{非合理バブル}
\begin{itemize}\setlength{\parskip}{0.5em}
\item
しかし、少数でも合理的な投資家がいれば株価はファンダメンタルバリューに等しくなるのではないか?
\item
実験の経験者が全くいないケース、経験者が3分の1いるケース、経験者が3分の2いるケース、全員経験者のケースで分けて実験を行ったところ、2〜4のケースは株価がファンダメンタルバリューに近づいた。
\item
つまり経験などを通じて合理性を備えた投資家が存在するならば、少数であっても効率的な価格付けがなされる。
\item
しかし現実にはすべての投資家が未経験の新しい状況も多い。(2000年ごろのITブームなど)
\end{itemize}
\end{frame}

\subsection{投機的バブル}
\begin{frame}
\frametitle{投機的バブル}
\begin{itemize}\setlength{\parskip}{0.5em}
\item
Hirota and Sunder(2007)は、最終期末の配当を得ることで利益を得る「長期の投資家市場」と、キャピタルゲインのみで利益を得る「短期の投資家市場」とにわけて実験を行う。
\item
長期の投資家市場は、最終期末の配当つまりファンダメンタルバリューの近辺で価格付けがなされた。一方で、短期の投資家市場は、今期の株価をもとに来期の株価を予想するのみとなりバブルが発生しやすかった。
\item
短期の投資家が支配的な市場においてはバブルが起きやすい。現実の市場においては、新興株、成長株、ハイテク株等にはバブルが起こりやすい。なぜなら、これらの株式は配当が通常は遠い将来に支払われるため、ほとんどの投資家が短期の投資家となるから。IT関連株もそう。
\end{itemize}
\end{frame}

\section{バブルを抑えるための方策}
\subsection{取引制度導入}
\begin{frame}
\frametitle{取引制度導入}
\begin{itemize}\setlength{\parskip}{0.5em}
\item
King, Smith, Williams and Van Boening(1993)は、今日やった実験に空売り、取引手数料、値幅制限等の制度を1つずつ導入してみた。
\item
しかし、結果は予想に反して株価にほとんど影響をあたえず、バブルの抑制に役立たなかった。
\item
これに対してHaruvy and Noussair(2006)は空売りの効果に関してバブルが抑制されるという実験結果を得ている。
\item
Sunder(1995)は実験で現物の市場に加えて先物市場を導入すると、現物市場の株価がファンダメンタルバリューに近づきバブルが消失すると発見。
\item
以上の実験結果は、有効な取引制度・環境の整備はバブル抑制につながりうるということを示す。
\end{itemize}
\end{frame}


\subsection{バブルへの処方箋}
\begin{frame}
\frametitle{バブルへの処方箋}
\begin{itemize}\setlength{\parskip}{0.5em}
\item
投資家へのファンダメンタルバリューの教育や証券アナリスト等の企業分析をより促進する政策(非合理バブルの抑制)
\item
長期の投資家を育成(投機的バブルの抑制)
\item
先物市場の整備など
\end{itemize}
\end{frame}

\section{今後の課題}
\subsection{卒論何しよう}
\begin{frame}
\frametitle{卒論何しよう}
\begin{itemize}\setlength{\parskip}{0.5em}
\item
最近とても貨幣の行く末について興味がある。(めちゃくちゃ話しが飛ぶが)\\
Bitcoin, ゲゼルマネー, 地域通貨, アジア共通通貨などなど。
\item
RubyでBitcoin実装とかいうサイトを見つけて取り組んでみてたものの、そもそもハッシュ関数とか公開鍵、暗号鍵みたいな暗号技術系の話しも含めてよくわからなかった。\\
理解できたらそのサイトの実装とかいうのをPythonで書き換えてみるのも面白そう。
\item
何か仮説をたてることが出来たら実験も織り交ぜたいかも。
\end{itemize}
\end{frame}

\section{参考文献}
\begin{frame}
\frametitle{参考文献}
\begin{itemize}\setlength{\parskip}{0.5em}
\item
『実験経済学の事始め』(松島斉,2007年)
http://www.econexp.org/hitoshi/07kotohajime.pdf
\item
『実験経済学入門』(濱口泰代,2010年)
http://www.econ.nagoya-cu.ac.jp/~yhamagu/20100930jikken1.pdf
\end{itemize}
\end{frame}




\end{document}